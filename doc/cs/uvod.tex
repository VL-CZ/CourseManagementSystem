\chapter{Úvod}

IT technologie pomáhají v každodenním životě s jeho organizací a dají se využít ke zvýšení
komfortu. Přehled informací o studiu, ať už se jedná o základní, střední, či vysokou školu, může výrazně ulehčit orientaci ve větším množství úkolů a studovaných předmětů.

Cílem práce je vytvořit webovou aplikaci pro správu různých typů výukových kurzů, jako například vysokoškolské přednášky a cvičení, předměty na základní či střední škole nebo také libovolný zájmový kurz. Program bude poskytovat rozhraní jak pro správce – učitele, tak pro uživatele – studenty. Bude umožňovat vytváření a správu kurzů. Správci kurzu budou moci měnit jeho obsah (přidávat, odebírat materiály, apod.) a zakládat nové úkoly a testy, které budou uživatelé následně vyplňovat. Otázky s nabídkou odpovědí bude systém vyhodnocovat automaticky, zbytek bude správce kurzu hodnotit ručně. Uživatel si pak bude moci v daném kurzu prohlédnout všechny známky, a také opravené testy a úkoly.

Alternativy jsou např. aplikace Bakaláři a Moodle. Na rozdíl od těchto programů nebude výsledná aplikace tak úzce zaměřená pro školy (bude ji možné snadno použít i např. na jazykové, zájmové kurzy, apod.).
Program je vyvíjen jako open-source, zdrojový kód se nachází ve veřejném repozitáři na Githubu.

Program bude fungovat jako single-page aplikace, s rozdělením na serverovou a klientskou část. 