%%% Fiktivní kapitola s ukázkami sazby

\chapter{Použité technologie}

V této kapitole se nachází popis technologií, použitých při vývoji aplikace.

\section{Serverová část}

\subsection{Jazyk C\#}
C\# je objektově orientovaný, staticky typovaný programovací jazyk, vyvinutý firmou Microsoft.
C\# podporuje koncepty zapouzdření, dědičnosti a polymorfismu. Programy v jazyce C\# běží na platformě .NET. 
Při kompilaci C\# programu je zdrojový kód nejprve zkompilován do mezikódu zvaného IL. 
Po spuštění programu pak modul CLR, který je součástí platformy .NET, provede JIT (just-in-time) kompilaci IL kódu do strojových instrukcí počítače.
Jazyk C\# lze využít k tvorbě konzolových aplikací, webových aplikací a stránek, formulářových aplikací ve Windows, softwaru pro mobilní zařízení, apod. 
\cite{CSharpDocs}

Jazyk C\# využíváme v serverové části aplikace. Používáme platformu .NET Core verze 3.1 a jazyk C\# verze 8.0.

\subsection{ASP .NET Core}
ASP.NET Core je open source framework, který slouží k vývoji webových aplikací na platformě .NET Core. Aplikace je možné psát v libovolném jazyce, který běží na platformě .NET Core (například v jazyce C\#). Jedná se o novější alternativu k frameworku ASP .NET. Součástí frameworku je mimo jiné webový server Kestrel a vestavěný IoC kontejner.
\cite{AspNetCoreDocs}

V aplikaci používáme framework ASP .NET Core verze 3.1 v projektu API. Tento projekt funguje jako REST API a slouží ke komunikaci serverové části s klientskou.

\subsection{Entity Framework Core}
Entity Framework Core je ORM framework. Objektově relační mapování je technika, která nám umožňuje objektově pracovat s daty v relační databázi. Databázové tabulky jsou ve frameworku reprezentované pomocí kolekcí, jednotlivé objekty v kolekci pak představují řádky v dané tabulce. Při práci s databází pak vůbec nepoužíváme jazyk SQL, pouze pracujeme s objekty a kolekcemi.
\cite{EfCoreDocs}

Tento framework využíváme v projektu Data, ve kterém se nachází objekty reprezentující databázové entity, a v projektu Services, který obsahuje služby pro komunikaci s databází.

\subsection{xUnit.net}
xUnit.net je framework, který slouží k testování aplikací na platformě .NET. Nejčastěji se používá k unit a integračním testům. 
\cite{xUnitDocs}

V aplikaci tento používáme xUnit.net v projektu TestEvaluation.Tests, který obsahuje unit testy tříd z projektu TestEvaluation.

\section{Klientská část}

\subsection{Jazyk TypeScript}
TODO: stručný popis
\cite{TypescriptDocs}

\subsection{Angular}
Pro frontend aplikace
\cite{AngularDocs}

TODO: stručný popis

\subsection{Bootstrap}
Stylování elementů.
\cite{BootstrapDocs}

\section{MSSQL databáze}
Databáze, ve které jsou uložena data
\cite{SqlServerDocs}

TODO: stručný popis

\section{Git}
Verzovací systém
\cite{GitDocs}

TODO: stručný popis, zmínit použití branches

\section{Github Actions}
CI tool.
\cite{GitHubActionsDocs}

TODO: stručný popis, zmínit použití