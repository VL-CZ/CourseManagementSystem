%%% Fiktivní kapitola s ukázkami sazby

\chapter{Použité technologie}

\section{Serverová část}

\subsection{Jazyk C\#}
C\# je objektově orientovaný, staticky typovaný programovací jazyk, vyvinutý firmou Microsoft.
C\# podporuje koncepty zapouzdření, dědičnosti a polymorfismu. Programy v jazyce C\# běží na platformě .NET. 
Při kompilaci je zdrojový kód nejprve zkompilován do mezikódu zvaného IL. 
Po spuštění programu pak modul CLR, který je součástí platformy .NET, provede JIT (just-in-time) kompilaci IL kódu do strojových instrukcí počítače.
Jazyk C\# lze využít k tvorbě konzolových aplikací, webových aplikací a stránek, formulářových aplikací ve Windows, softwaru pro mobilní zařízení, apod. \cite{CSharpDocs}

Jazyk C\# využíváme v serverové části aplikace. Používáme platformu .NET Core verze 3.1 a jazyk C\# verze 8.0.

\subsection{ASP .NET Core Web API}
Pro backend aplikace \cite{AspNetDocs}

TODO: popis frameworku

\subsection{Entity Framework Core}
ORM pro práci s databází

TODO: popis frameworku

\subsection{xUnit}
Framework pro unit testy

TODO: popis frameworku

\section{Klientská část}

\subsection{Jazyk TypeScript}
TODO: stručný popis

\subsection{Angular}
Pro frontend aplikace

TODO: stručný popis

\section{MSSQL databáze}
Databáze, ve které jsou uložena data

TODO: stručný popis

\section{Git}
Verzovací systém

TODO: stručný popis, zmínit použití branches

\section{Github Actions}
CI tool.

TODO: stručný popis, zmínit použití