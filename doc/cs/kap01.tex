%%% Fiktivní kapitola s ukázkami sazby

\chapter{Popis a analýza použitých technologií}

V této kapitole se nachází popis technologií, použitých při vývoji aplikace.

Program je ve formě single-page webové aplikace, což je v dnešní době velmi rozšířený typ webových aplikací.

Každá webová aplikace je typicky rozdělená na dvě části: serverovou a klientskou část. Serverová část aplikace běží na serveru a často má na starosti komunikaci s databází a vnitřní logiku aplikace. Klientská část běží naopak v prohlížeči uživatele a často bývá naprogramovaná v jazyce JavaScript. 

Klasické webové aplikace mají typicky velkou většinu kódu v serverové části. Při dotazu na danou URL server požadavek zpracuje, a vrátí uživateli vygenerované HTML.

Single-page aplikace přesouvají část kódu do klientské části a dotazy na server, který typicky funguje jako API, probíhají ve většině případů asynchronně na pozadí.
Server vrací data ve formátu JSON, jež jsou následně zpracovány klientem. Oproti klasickým webovým aplikacím tedy není potřeba při dotazu na data stránku znovu načítat.
Single-page aplikace nám umožňují oddělení serverové a klientské části, každá část může být napsána v jiném programovacím jazyce.

Velkou výhodou webových aplikací je jejich přenositelnost, jelikož klientská část typicky může běžet v libovolném prohlížeči.
Další výhodou je jednoduchost použití, uživatel nemusí program stahovat ani instalovat, stačí pouze mít webový prohlížeč.

V následujících podkapitolách se nachází popis použitých technologií a uvedení případných alternativ.

\section{Serverová část}

\subsection{Jazyk C\#}
C\# je objektově orientovaný, staticky typovaný programovací jazyk, vyvinutý firmou Microsoft.
C\# podporuje koncepty zapouzdření, dědičnosti a polymorfismu. Programy v jazyce C\# běží na platformě .NET. 
Při kompilaci C\# programu je zdrojový kód nejprve zkompilován do mezikódu zvaného CIL. 
Po spuštění programu pak modul CLR, který je součástí platformy .NET, provede JIT (just-in-time) kompilaci CIL kódu do strojových instrukcí počítače.
Jazyk C\# lze využít k tvorbě konzolových aplikací, webových aplikací a stránek, formulářových aplikací ve Windows, softwaru pro mobilní zařízení, apod. 
\cite{CSharpDocs}

Jazyk C\# využíváme v serverové části aplikace. Používáme platformu .NET Core verze 3.1 a jazyk C\# verze 8.0.

\subsection{ASP .NET Core}
ASP.NET Core je open source framework, který slouží k vývoji webových aplikací na platformě .NET Core. Aplikace je možné psát v libovolném jazyce, který běží na platformě .NET Core (například v jazyce C\#). Jedná se o novější alternativu k frameworku ASP .NET. Součástí frameworku je mimo jiné webový server Kestrel a vestavěný IoC kontejner.
\cite{AspNetCoreDocs}

V aplikaci používáme framework ASP .NET Core verze 3.1 v projektu API. Tento projekt funguje jako REST API a slouží k obsluze HTTP požadavků.

\subsection{Entity Framework Core}
Entity Framework Core je ORM framework. Objektově relační mapování je technika, která nám umožňuje objektově pracovat s daty v relační databázi. Framework reprezentuje databázové tabulky pomocí kolekcí, jednotlivé objekty v kolekci pak představují řádky v dané tabulce. Při práci s databází pak vůbec nepoužíváme jazyk SQL, pouze pracujeme s objekty a kolekcemi.
\cite{EfCoreDocs}

Tento framework využíváme v projektu Data, ve kterém se nachází objekty reprezentující databázové entity, a v projektu Services, který obsahuje služby pro komunikaci s databází.

\subsection{xUnit.net}
xUnit.net je framework, který slouží k testování aplikací na platformě .NET. Nejčastěji se používá k unit a integračním testům. 
\cite{xUnitDocs}

V aplikaci tento používáme xUnit.net v projektu TestEvaluation.Tests, který obsahuje unit testy tříd z projektu TestEvaluation.

\subsection{Alternativy}

K vývoji serverové části aplikace můžeme použít velké množství jiných technologií, jelikož většina populárních programovacích jazyků nabízí nějakou knihovnu, příp. framework pro tvorbu webových aplikací.  Jako příklad můžeme uvést jazyk Java v kombinaci s frameworkem Spring, PHP nebo Python s knihovnou Flask. 

Jazyk C\# s použitím frameworku ASP .NET Core jsem zvolil hlavně z toho důvodu, že se jedná o moderní staticky typovaný jazyk, který považuji za nejlepší volbu z výše uvedených. S vývojem v tomto jazyce mám také nejvíce zkušeností. 

Pro práci s databází existuje v jazyce C\# velké množství knihoven a frameworků. Kromě již zmíněného frameworku Entity Framework Core můžeme uvést také knihovnu ADO.NET, micro ORM Dapper a framework NHibernate. 
Tyto technologie se můžou lišit podle míry abstrakce nad databází, např. při použití frameworku Dapper se dotazujeme pomocí klasického SQL, naopak Entity Framework Core nám umožňuje s tabulkami pracovat stejně jako s kolekcemi.

Entity Framework Core je v současnosti pravděpodobně nejpoužívanější z těchto technologií, a umožňuje nám pracovat s databází velmi pohodlně, z tohoto důvodu jsem jej tedy v aplikaci použil. 

\section{Klientská část}

\subsection{Jazyk TypeScript}
TypeScript je programovací jazyk vyvinutý firmou Microsoft. Jedná se o nadstavbu jazyka JavaScript, oba jazyky tedy používají stejnou syntaxi. Kód v jazyce TypeScript se kompiluje do JavaScriptu.
Oproti jazyku JavaScript používá statické typování a umožňuje používat třídy, rozhraní a další konstrukce z OOP. 
\cite{TypescriptDocs}

Téměř celá klientská část aplikace (kromě HTML šablon a CSS stylů) je napsána v jazyce TypeScript.

\subsection{Angular}
Angular je frontend framework pro tvorbu single-page webových aplikací. Framework používá jazyk TypeScript a jeho architektura je založená na komponentách.
Komponenty jsou jednotky kódu, ze kterých se skládá UI aplikace. Každá komponenta obsahuje TypeScript třídu označenou dekorátorem \textit{@Component()}, HTML šablonu, a případně i soubor s CSS styly. 
Součástí frameworku je i vestavěný HTTP klient pro snadnější komunikaci pomocí HTTP, router sloužící k navigaci mezi stránkami a velké množství dalších knihoven. Angular nám umožňuje používat obousměrný data binding.
\cite{AngularDocs}

Tento framework využíváme v klientské části aplikace, celá klientská část programu je Angular projekt.

\subsection{Bootstrap}
Bootstrap je CSS framework, který se používá k designu webových stránek.
Umožňuje jednoduše vytvářet responzivní layout a obsahuje velké množství definovaných CSS stylů, které lze použít ke stylování HTML elementů.
\cite{BootstrapDocs}

Bootstrap hojně používáme ke stylování HTML elementů v šablonách Angular komponent.

\subsection{Alternativy}
K vývoji klientské části aplikace se nejčastěji používá jazyk JavaScript, případně jeho nadstavba TypeScript. Mezi nejpoužívanější frameworky patří v současnosti Angular, React a Vue.js. 
Poměrně novou možností je technologie WebAssembly, zde můžeme uvést například framework Blazor, jež nám umožňuje programovat klientskou část aplikace v jazyce C\#.

Tyto technologie velmi zjednodušují tvorbu moderních a přívětivých uživatelských rozhraní. 
Ve většině případů tyto frameworky skládají uživatelské rozhraní ze znovupoužitelných částí, kterým říkáme komponenty.

Framework Angular považuji za nejlepší volbu z těchto technologií. Jedná se o moderní a prověřenou technologii, jež používá jazyk TypeScript. TypeScript je oproti JavaScriptu statický typovaný, což je dle mého názoru velká výhoda. 

Další výhodou frameworku je velké množství knihoven, které můžeme v aplikaci jednoduše použít, jako příklad můžeme uvéest vestavěného HTTP klienta či Router. 
S vývojem aplikací ve frameworku Angular mám již také zkušenosti, což je další z důvodů, které vedly k volbě tohoto frameworku.

\section{Microsoft SQL Server}
Microsoft SQL Server je multiplatformní relační databázový systém. Ke komunikaci s databází využívá jazyk T-SQL, což je rozšíření klasického SQL. K administraci Microsoft SQL Serveru a práci s databázemi můžeme použít například aplikaci SQL Server Management Studio.
\cite{SqlServerDocs}

Microsoft SQL Server používáme jako databázi pro data v naší aplikaci. Jazyk T-SQL vůbec nepoužíváme, jelikož s daty pracujeme pomocí ORM frameworku Entity Framework Core.

\section{Git}
Git je distribuovaný systém správy verzí, který se velmi často používá při vývoji softwaru. Podporuje tzv. větvení, což znamená, že se můžeme odloučit od hlavní linie vývoje a pokračovat ve vývoji v nové větvi, aniž bychom do původní větve zasahovali. Pomocí příkazu \textit{merge} pak můžeme změny promítnout zpět do hlavní větve.
\cite{GitDocs}

Při vývoji aplikace používáme několik větví: \textit{master}, \textit{develop} a \textit{feature} větve.
\begin{itemize}
	\item \textit{Feature} větve používáme při programování nové funkcionality. Typicky má\-me pro každou funkcionalitu samostatnou větev, jež se ve většině případů jmenuje \textit{feature/popis-funkcionality}.
	\item \textit{Develop} je větev, do které se provádí \textit{merge} z \textit{feature} větví. Při vývoji nové funkcionality tedy nejprve vytvoříme novou větev a po dokončení vývoje provedeme \textit{merge} do větve \textit{develop}.
	\item \textit{Master} větev obsahuje verzi aplikace, která je plně funkční a lze spustit. V případě, že dokončíme nějakou ucelenou část aplikace a ověříme, že funguje správně, můžeme provést \textit{merge} z větve \textit{develop} do \textit{master}.
\end{itemize}
Díky tomuto rozdělení větví můžeme pohodlně pracovat na více částech aplikace zároveň, a máme stále aspoň jednu verzi aplikace, která je plně funkční (ve větvi \textit{master}).

Aplikace je uložená ve veřejném GitHub repozitáři na této URL: \url{https://github.com/VL-CZ/CourseManagementSystem}.

\section{GitHub Actions}
Technologie GitHub Actions slouží k automatizaci úloh během životního cyklu vývoje softwaru. Používají tzv. CI Workflows, což je automatizovaná procedura, která je typicky spuštěna nějakou akcí (např. přidáním nového commitu).
Ke konfiguraci Workflows se používají soubory ve formátu YAML, které se nachází ve složce \textit{.github/workflows}.
\cite{GitHubActionsDocs}

V této aplikaci používáme dva Workflows, které se spustí po každém commitu ve větvích \textit{master} a \textit{develop}.
\begin{itemize}
	\item .NET Workflow - provede build serverové části aplikace a spustí všechny testy
	\item Angular Workflow - provede build klientské části aplikace
\end{itemize}

Použitím technologie GitHub Actions si můžeme lehce ověřit, že přidáním nové funkcionality jsme nerozbili nějakou již existující část aplikace. Po přidání nových změn si můžeme jednoduše ověřit, že Workflows proběhly úspěšně.
