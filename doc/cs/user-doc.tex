\section{Uživatelská dokumentace}

V této kapitole se nachází uživatelská dokumentace aplikace.
Popis všech funkcí + screenshoty?

TODO: rozepsat mnohem více do detailu (krok po kroku + snímky obrazovky)

(převzato ze specifikace ročníkového projektu)

Na obrázku (screenshot) můžeme vidět uživatelské rozhraní aplikace.

\subsection{Registrace nového uživatele}

\begin{itemize}
	\item Pro registraci nového uživatele nejprve klikneme na tlačítko Register v horním menu.
	\item Uživatelé v aplikaci jsou identifikováni pomocí e-mailu. Do registračního formuláře zadáme tedy e-mail a heslo nového uživatele, a stiskneme tlačítko Register.
	\item Zobrazí se stránka s potvrzením registrace. Vzhledem k našim potřebám aktuálně neprovádíme ověření e-mailu. Pokud bychom chtěli tuto funkcionalitu přidat, můžeme postupovat například podle návodu v dokumentaci. \cite{AspNetCoreDocs} Klikneme na odkaz s textem 'Click here to confirm your account', a tím je registrace dokončena. Pomocí odkazu pak můžeme přejít na stránku s přihlášením.
\end{itemize}

\subsection{Přihlášení uživatele}

\begin{itemize}
	\item Pro přihlášení uživatele nejprve stiskneme tlačítko Login v horním menu.
	\item Zobrazí se přihlašovací formulář, do kterého vyplníme e-mail a heslo. (screenshot)
	\item Po odeslání formuláře nás aplikace přihlásí a přesměruje na stránku se seznamem kurzů.
	\item Pokud se následně chceme odhlásit, vybereme v menu volbu Logout.
\end{itemize}

\subsection{Zobrazení stránky se seznamem kurzů}

\begin{itemize}
	\item Po kliknutí na odkaz Courses v menu se zobrazí stránka, na které se nachází seznam kurzů. (screenshot)
	\item Na této stránce se nachází několik sekcí. Sekce Member courses obsahuje všechny kurzy, jejichž jsme členy. Po kliknutí na odkaz s názvem kurzu se dostaneme na stránku s detaily kurzu.
	\item Sekce Managed courses obsahuje všechny kurzy, které spravujeme (tzn. jsme administrátoři). Kliknutím na odkaz s názvem kurzu se opět dostaneme na stránku s detaily kurzu. Každý z těchto kurzů můžeme také smazat -- kliknutím na tlačíko "Delete" vedle názvu kurzu.
	\item Dále se na této stránce nachází sekce, které slouží k přidání nového kurzu a zápisu do kurzu (více viz. dále). Úplně dole je pak sekce, ve které můžeme vidět náš uživatelský identifikátor.
\end{itemize}

\subsection{Vytvoření nového kurzu}

\begin{itemize}
	\item Na stránce se seznamem kurzů je sekce s názvem Add new course, která obsahuje formulář pro přidání nového kurzu.
	\item Do políčka Name vyplníme jméno kurzu a klikneme na tlačítko Add.
	\item Vytvoří se nový kurz, který má právě jednoho administrátora -- nás.
\end{itemize}

\subsection{Zápis do kurzu}

\begin{itemize}
	\item Nejprve přejdeme na stránku se seznamem kurzů a vybereme sekci Enroll to a course.
	\item K přihlášení do kurzu je nejprve třeba znát jeho identifikátor, ten nám může například zaslat nějaký z administrátorů. Po získání identifikátoru jej vložíme do políčka s názvem 'Course ID' a formulář odešleme kliknutím na tlačítko Enroll.
	\item Tímto vytvoříme žádost o přihlášení do daného kurzu. Členem kurzu se staneme až poté, co nám žádost schválí některý z administrátorů.
\end{itemize}

\subsection{Zobrazení detailu kurzu}

\begin{itemize}
	\item Po kliknutí na kurz v seznamu kurzů se dostaneme na stránku s detaily.
	\item Úplně nahoře vidíme sekci s názvem kurzu (screenshot), ve které se zároveň nachází identifikátor tohoto kurzu, který se používá při zápisu do kurzu. Členové kurzu zde také vidí tlačítko s textem 'My Grades', pomocí kterého se dostanou na stránku s hodnocením studenta.
	\item Pod touto sekcí se nachází seznam testů, rozdělený do tří kategorií:
		\begin{itemize}
			\item Not Published -- tyto testy zatím nebyly publikované, mohou si je zobrazit pouze administrátoři.
			\item Active -- tyto testy jsou aktivní, aktuálně je můžeme odevzdávat.
			\item After deadline -- testy, které již není možné odevzdat.
		\end{itemize}
		Pokud některá z kategorií neobsahuje žádný test, tak se na stránce vůbec nezobrazuje. Studenti kurzu vidí pouze kategorii 'Active', zatímco administrátorům se zobrazují všechny tři výše uvedené kategorie.
		
		U každého testu je uvedena informace, jestli je hodnocený (GRADED, příp. NOT GRADED) a případně i deadline.
		Po kliknutí na název testu se student dostane buď na stránku s odevzdáním, anebo na stránku s vyhodnoceným řešením, pokud již test odevzdal. Administrátory kurzu aplikace přesměruje na stránku s detaily testu.
		
	\item Na stránce je dále sekce se sdílenými soubory, které si můžeme stáhnout. Správci kurzu zde vidí i formulář pro nahraní nového souboru.
	\item Dále se zde nachází sekce se seznamem členů a administrátorů daného kurzu, které se zobrazují pouze administrátorům (více viz. dále).
	\item Na stránce dále můžeme vidět fórum s příspěvky. U každého příspěvku je uvedený text a jeho autor. Správci kurzu zde vidí také tlačítko pro smazání příspěvku.
	\item Pod fórem se nachází formulář, pomocí kterého lze přidávat nové příspěvky.
	\item Úplně dole je pak tlačítko, které slouží k opuštění kurzu. Tato akce je nevratná, pokud tedy kurz opustíme, nemůžeme původní účet v aplikaci nijak obnovit. Můžeme ovšem zažádat o nové zapsání do kurzu.
\end{itemize}

\subsection{Zobrazení detailu studenta}

\begin{itemize}
	\item Na stránku s detaily studenta se můžeme dostat ze stránky s detaily kurzu. Studenti si stránku zobrazí pomocí tlačítka 'My Grades', administrátoři se sem dostanou pomocí kliknutí na studenta v sekci s členy kurzu.
	\item Na obrázku (screenshot), vidíme, že na této stránce se nachází komponenty s hodnocením daného studenta. 
	\item Nachází se zde tabulka s odevzdanými testy, po kliknutí na název testu se dostaneme na stránku s vyhodnoceným řešením. U každého testu pak vidíme získané skóre, váhu testu (tedy dopad na celkovou známku), datum a čas odevzdání. Poslední sloupec nám říká, jestli bylo odevzdané řešení již revidované
	\item Pod seznamem odevzdaných testů se nachází komponenta s dalšími známkami. Tyto známky nepatří k žádnému testu. 
	\item Na stránce se dále nachází sekce s celkovým skóre studenta, ve které můžeme vidět vážený průměr všech známek dané osoby.
	\item Pod touto sekcí se nachází formulář pro přidávání známek, který se zobrazuje pouze administrátorům.
	\item Ve spodní části stránky je komponenta s odevzdanými kvízy (tzn. nehodnocenými testy). Po kliknutí na název kvízu nás aplikace přesměruje na detail odevzdaného řešení.
\end{itemize}

\subsection{Odevzdání testu}

\begin{itemize}
	\item Nejprve přejdeme na stránku nějakého kurzu, jehož jsme členy.
	\item V seznamu testů vybereme ten, který chceme odevzdat.
	\item Pokud jsme tento test zatím neodevzdali, aplikace nás přesměruje na stránku s odevzdáním daného testu (screenshot). V případě, že jsme test již odevzdali, zobrazí se stránka s vyhodnoceným řešením.
	\item V horní části vidíme název testu, deadline -- čas, do kterého je třeba test odevzdat a několik dalších informací. Test následně vyplníme, u některých otázek máme na výběr z možností odpovědí.
	\item Stisknutím tlačítka Save answers dojde k uložení odpovědí. Můžeme tedy stránku opustit, příp. zavřít a při dalším načtení stránky se načtou i uložené odpovědi.
	\item Odevzdání testu provádíme pomocí tlačítka Submit. Tato akce je nevratná, po odevzdání testu jej není možné nijak upravovat ani znovu odevzdat.
	\item Po odevzdání testu nás aplikace přesměruje na stránku s vyhodnoceným testem.
\end{itemize}

\subsection{Zobrazení vyhodnoceného testu}

\begin{itemize}
	\item 
\end{itemize}

\subsection{Postranní menu s názvem kurzu}



\subsection{Administrace členů kurzu}

\subsection{Sdílení souboru v kurzu}

\subsection{Potvrzení / zamítnutí žádosti o zápis do kurzu}

\subsection{Vytvoření nového testu}

\subsection{Editace testu}

\subsection{Publikace testu}

\subsection{Odstranění testu}

\subsection{Oprava testu}

\subsection{Vytvoření nové známky}

Administrátoři kurzu budou moci vytvářet testy, které budou žáci vyplňovat. V případě otázek s nabídkou odpovědí systém správnost sám vyhodnotí, otázky s volnou odpovědí budou hodnoceny ručně. Každý uživatel pak v daném kurzu uvidí seznam známek a bude si moci opravený test zobrazit a prohlédnout. Aplikace bude rozlišovat mezi testy a kvízy (kvízy se nebudou hodnotit známkou).

Uživatelé budou moci prostřednictvím aplikace odevzdávat úkoly, které jim budou poté ohodnoceny.

Administrátoři kurzu budou moci editovat obsah kurzu (např. přidávat nové materiály, soubory, apod.) 

U každého kurzu bude k dispozici fórum. 

Uživatelé si budou moci vytvářet nové účty, správci je budou moci do daného kurzu zapsat.
