\section{Uživatelská dokumentace}
Popis všech funkcí + screenshoty?

TODO: rozepsat mnohem více do detailu (krok po kroku + snímky obrazovky)

(převzato ze specifikace ročníkového projektu)

\subsection{Vytvoření a administrace kurzu}
Uživatelé budou moci vytvářet nové kurzy, administrátoři daného kurzu budou moci přidávat a odebírat členy. Každý uživatel uvidí v aplikaci seznam kurzů, kam je přihlášen.

\subsection{Vytváření a vyplňování testů}
Administrátoři kurzu budou moci vytvářet testy, které budou žáci vyplňovat. V případě otázek s nabídkou odpovědí systém správnost sám vyhodnotí, otázky s volnou odpovědí budou hodnoceny ručně. Každý uživatel pak v daném kurzu uvidí seznam známek a bude si moci opravený test zobrazit a prohlédnout. Aplikace bude rozlišovat mezi testy a kvízy (kvízy se nebudou hodnotit známkou).

\subsection{Odevzdávání a hodnocení úkolů}
Uživatelé budou moci prostřednictvím aplikace odevzdávat úkoly, které jim budou poté ohodnoceny.

\subsection{Správa obsahu kurzu}
Administrátoři kurzu budou moci editovat obsah kurzu (např. přidávat nové materiály, soubory, apod.) 

\subsection{Fórum k danému kurzu}
U každého kurzu bude k dispozici fórum. 

\subsection{Vytváření a správa účtů}
Uživatelé si budou moci vytvářet nové účty, správci je budou moci do daného kurzu zapsat.
